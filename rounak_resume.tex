% LaTeX file for resume 
% This file uses the resume document class (res.cls)

\documentclass{res} 
\usepackage[margin=0.3in]{geometry}
\usepackage{hyperref}
\topmargin=-0.5in
% the margin option causes section titles to appear to the left of body text 
\textwidth=7.2in % increase textwidth to get smaller right margin
%\usepackage{helvetica} % uses helvetica postscript font (download helvetica.sty)
%\usepackage{newcent}   % uses new century schoolbook postscript font 


\begin{document} 
	\renewcommand{\namefont}{\bfseries\LARGE}
	\renewcommand{\sectionfont}{\bfseries\Large}
	
 
\name{Rounak Chawla\\[12pt]} % the \\[12pt] adds a blank line after name
 
\address{ +1 2163032983 \\  rounak@case.edu}
\address{\hfill \href{https://github.com/theroshogolla}{Github: theroshogolla} \\ \hfill \href{https://www.linkedin.com/in/rounakchawla}{LinkedIn: rounakchawla}}
 
\begin{resume} 
 
\section{Education} 
M.S. Candidate in Computer Science, Case Western Reserve University (CWRU), Cleveland, OH, May 2025 \\
B.S. in Computer Science with Minors in Applied Data Science and Philosophy, CWRU, May 2023 \\
Specialization: Artificial Intelligence \\
Major GPA: 3.0 \\
International Baccalaureate, UWC Mahindra College, Mulshi, India,  May 2018

\section{Work Experience} 

{\large{\bf GAIT Labs, Hyderabad, India} – LLM Engineering Intern \hfill October 2024 - Present}
\begin{itemize} \itemsep -2pt
	\item Developing a plug-and-play \textbf{RAG pipeline} that is integrable with \textbf{HuggingFace Transformers} and \textbf{Ollama models}, for financial reasoning. Automated the parsing of financial documents with \textbf{LlamaIndex} and \textbf{PyMuPdf}. Deployed the parallel pipeline to multiple \textbf{AWS EC2} instances.
	\item Developed a systematic experimental pipeline using \textbf{Pandas} and \textbf{HuggingFace Transformers} to evaluate the performance of various language models on \textbf{sequence summarization} on financial documents.
	\item Finetuned several LLM architectures on financial datasets using \textbf{torchtune}.
\end{itemize}

{\large{\bf CWRU SDLE Laboratory, Cleveland, OH} – Graduate Researcher \hfill January 2021 - Present}
\begin{itemize} \itemsep -2pt
	\item Leading the development and maintenance of the lab's HPC and distributed systems by building and deploying \textbf{Docker containers}, maintaining a shared dependency environment of \textbf{Python} and \textbf{Linux} packages, developing \textbf{database schemas} for complex scientific data, and collaborating with external researchers, vendors, and university stakeholders.
	\item Developed a portable, containerized instantiation of a hybrid HPC/distributed computing cluster using \textbf{Docker Compose}, encapsulating several key infrastucure and ML services such as \textbf{Slurm, HDFS, YARN, Spark, Open OnDemand, Tensorflow, PyTorch, and R}, serving as a test-bed for cross-framework workflows and enabling key government collaborators to securely interact with the lab's entire software stack locally.
	\item Built a distributed data pipeline on an \textbf{Apache Hadoop} cluster for the secure long term storage and processing of multimodal data and files, leveraging tools such as \textbf{Apache Ozone, Apache Spark, Apache Impala,} and \textbf{Kerberos}. Optimized existing ETL Spark jobs leading to an 8x improvement in execution time.
	\item Developed a package that provides a scaleable \textbf{Python} and \textbf{R} interface to the \textbf{Slurm} job scheduler, allowing users to submit fleets of machine learning jobs to a high performance computing (HPC) cluster. Enabled researchers to train up to 350,000 models in about 15 minutes.
	\item Trained and fine-tuned deep learning models for computer vision, using the \textbf{YOLOv7} and \textbf{U-Net} architectures in \textbf{Tensorflow} and \textbf{PyTorch} . Optimized preprocessing and training for \textbf{Nvidia GPUs} using the \textbf{TensorBoard Profiler}.
	\item Interfacing with the university's HPC team to troubleshoot bugs and outages.
\end{itemize}


 {\large{\bf HotSpot Inc, Seattle, WA} – Fullstack Intern   \hfill June 2020 - September 2020}
 \begin{itemize} \itemsep -2pt  % reduce space between items
 \item Directed and developed a data visualization dashboard for timestamped capacitive sensing data, using \textbf{ReactJS}, \textbf{Flask}, and \textbf{matplotlib}, for an IoT appliance startup.
 \item Constructed an optimized \textbf{NoSQL} time series data pipeline using \textbf{AWS DynamoDB,} the \textbf{boto3} Python package, and the \textbf{AWS SDK for JavaScript.}
 \item Redesigned the product UI in \textbf{ReactJS} with \textbf{TypeScript} and \textbf{CSS}.
 \end{itemize}

 {\large{\bf Boundary Labs, Cleveland, OH –} Backend and Hardware Engineer Intern  \hfill April  2019 - November 2019 }
 \begin{itemize} \itemsep -2pt  % reduce space between items
 \item A \textbf{Y Combinator} Winter 2020 company, Boundary Labs (now \textbf{Workbench Technologies}) was an IoT startup developing edge devices for utilization and efficiency monitoring of factory machines 
 \item Spearheaded a hardware and software solution for continuous amperage monitoring, at a rate of 2000 samples per second, using \textbf{Arduino} and multithreaded \textbf{Python} scripts on \textbf{Raspberry Pis}, and cloud storage on \textbf{Amazon S3}.
 \item Constructed a backend architecture and data pipeline to store and query device data (which  handled data for 15 IoT devices) using \textbf{PostgreSQL}, \textbf{AWS RDS}, and  \textbf{boto3.} 
 \item Implemented an \textbf{Android} application in \textbf{Java} that served as a 24/7 digital endpoint for manufacturing operators, for machine status monitoring and data entry. Used on 3 machine lines.
 \end{itemize}

{\large{\bf Skylark Drones, Bengaluru, India –} Drone Electronics Intern  \hfill  June 2017 - July 2017}
\begin{itemize} \itemsep -2pt %reduce space between items
	\item Researched  the ground control systems \textbf{Mission Planner} and \textbf{MAVProxy}, and the \textbf{MAVLink protocol}, and charted improvements to the indigenous software of India's largest drone services company.
	\item Developed a Python script for post-processing video metadata to assist in the survey of roadways.
\end{itemize}

\section{Publications}
\begin{itemize}
	\item A. Nihar et al., "Accelerating Time to Science using CRADLE: A Framework for Materials Data Science," 2023 IEEE 30th International Conference on High Performance Computing, Data, and Analytics (HiPC), Goa, India, 2023, pp. 234-245, doi: 10.1109/HiPC58850.2023.00041.
	
	\item T. G. Ciardi et al., “Materials data science using cradle: A distributed, data-centric approach,” MRS Communications, Jul. 2024. doi:10.1557/s43579-024-00616-6 
\end{itemize}

\section{Awards, Projects, Activities, and Leadership} 
{\large{\bf GraphMaster} – Project \hfill March 2023 - Present}
\begin{itemize} \itemsep -2pt
	\item Developed a graph-based representation for chess games.
	\item Trained various Graph Neural Network architectures (\textbf{GCNs, GraphSAGE, GRCNs}) in \textbf{PyTorch} for win prediction of a chess game from a given position.
\end{itemize}
{\large{\bf LiON Care, Cleveland, OH} – Team Lead \hfill October 2020 - June 2021}
\begin{itemize} \itemsep -2pt
	\item Led a 5-person team dedicated to building software to predictively monitor the health of lithium-ion batteries.
	\item Built a Gaussian Process Regression Model to predict battery aging using \textbf{Python sklearn} and \textbf{AWS}.
	\item Planned and conducted interviews with 30 prospective customers as part of the NSF's I-Corps Program.
\end{itemize}
{\large{\bf  ThinkEnergy Fellowship } – Case Western Reserve University  \hfill    August 2020 - May 2021 }
\begin{itemize} \itemsep -2pt
	\item Awarded a research fellowship focused on technology development, entrepreneurship, and policy in the energy sector.
\end{itemize}
{\large{\bf  CWRU Undergraduate Diversity Collaborative} – Vice President of Finance     \hfill    April 2019 - April 2020 }
 \begin{itemize} \itemsep -2pt
 \item Coordinated finances for over 30 different cultural and diversity student organizations on the CWRU campus 
 \end{itemize}

		

\end{resume} 
\end{document} 



